\section{Elements of the program DCIP2D}
\label{Elements}

\subsection{Introduction}

The \codeName{\prog} program library consists of the programs:
\begin{enumerate}
\item \codeName{DCIPF2D}: performs forward modelling for DC and IP data 
\item \codeName{DCINV2D}: inverts DC potentials to recover a conductivity model 
\item \codeName{IPINV2D}: inverts IP data to recover a chargeability model 
\end{enumerate}

Each of the above programs requires input files as well as the specification of parameters in order to run. However, some files are used by a number of programs. Before detailing the run procedures for each of the above programs we first present information about these general files. 

\subsection{General files for \prog~programs}

There are six general files which are used in \prog. These are: 

\begin{enumerate}
\item \fileName{observation}: specifies the observed measurements and the associated electrode locations
\item \fileName{electrodes}: contains the electrode locations for forward modelling
\item \fileName{mesh}: contains the finite difference mesh for the 2D modelling and inversions 
\item \fileName{topography}: contains the topographic data 
\item \fileName{model}: structure to hold cell values for models: conductivity, chargeability or active
\item \fileName{weighting}: file that contains special weightings which alter the type of model produced in the inversions
\item \fileName{active}: a special type of model file specifying the active cells
\end{enumerate}

% Observation file description
\input observations.tex

% Electrode locations file description
\input locations.tex

% 2D mesh file description
\input mesh2d.tex

% 2D topography file description
\input topo2d.tex

% 2D model file description
\input model2d.tex

% 2D weights file description
\input weights2d.tex